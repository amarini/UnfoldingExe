\documentclass[a4paper,11pt]{article}

\title{Exercize on unfolding distributions}
\author{ACM}
\date{\today}

%%
\usepackage{bbding}
\usepackage{ifthen}
\usepackage{graphicx}
\usepackage{amsmath}
%
\setlength{\parindent}{0pt}

\newcommand{\difficulty}[1]
                {
                \vspace{-30pt}
                \begin{flushright}
                \mbox{
                \ifthenelse{ #1 = 1 }{\FiveStar \FiveStarOpen \FiveStarOpen \FiveStarOpen \FiveStarOpen} {}
                \ifthenelse{ #1 = 2 }{\FiveStar \FiveStar \FiveStarOpen \FiveStarOpen \FiveStarOpen} {}
                \ifthenelse{ #1 = 3 }{\FiveStar \FiveStar \FiveStar \FiveStarOpen \FiveStarOpen} {}
                \ifthenelse{ #1 = 4 }{\FiveStar \FiveStar \FiveStar \FiveStar \FiveStarOpen} {}
                \ifthenelse{ #1 = 5 }{\FiveStar \FiveStar \FiveStar \FiveStar \FiveStar} {}
                }
                \end{flushright}
                }




\usepackage{placeins}
\usepackage{amsmath,amssymb,amsfonts}
%%
\newcommand{\meas} {\ensuremath \vec{y}}
\newcommand{\truth}{\ensuremath \vec{x}_\textup{T}}
\newcommand{\resp} {\ensuremath \mathbf{R}}
\newcommand{\back} {\ensuremath \vec{b}}

\newcommand{\respt}{\ensuremath \mathbf{\tilde{R}}}
\newcommand{\strength}{\ensuremath \vec{\mu} } 

\newcommand{\mc}{\ensuremath \vec{x}_\textup{MC}}
%% I version
\newcommand{\truthI}[1]{\ensuremath x^\textup{T}_{#1}}
\newcommand{\strengthI}[1]{\ensuremath \mu_{#1} }
\newcommand{\mcI}[1]{\ensuremath x^\textup{MC}_{#1}}
\newcommand{\respI}[2]{\ensuremath R_{#1,#2} }
\newcommand{\resptI}[2]{\ensuremath \tilde{R}_{#1,#2} }

%%%  Draft
\newcommand{\fixme}[1]{ \mbox{\bf{FIXME:} \it{#1}} } 

\begin{document}
\maketitle
\section{Introduction}
Unfolding is the ``art'' of being able to undo detector effects, such as smearings, efficiencies or background contaminations.
Unfolding techniques refers to a set of statistical tools explained in many books such as Ref.~\cite{Cowan}, 
or summurized in short letter and articles \cite{Cowan:unfolding}. 
Different software suites provede an implementation of such methods; 
in particle physics one of the most spread is {\scshape RooUnfold} \cite{RooUnfold}, 
which can be used within the {\scshape root} software. \fixme{citation needed}

The basic idea behind unfolding is the fact that (at least on certain observables) detector acts linearly on the shapes; this is the case for example for cross-section measurements, where the observables is the total number of event registered in the detector.
The equation we will encounter is the following:
\begin{equation}
	\meas= \respt \cdot \truth + \back
	\label{eqn:base}
\end{equation}
it means that the measured spectra ($\meas$) can be obtained by a smearing and efficiency ($\respt$) applied on the truth distribution ($\truth$). To that a possible background ($\back$) can be added to the measured distribution; this background can come from different sources, like different processes, but also different region of the phase-space where the smearings effects can lead to migration inside the region of interest.

Equation~\ref{eqn:base} can come in different shapes, one of the most relevant is the rescaling of it with respect to a particular truth/prediction \cite{SVD}:
\begin{equation}
	\meas = \resp \cdot \strength + \back
\end{equation}
where $\strengthI{i} = \frac{\truthI{i}}{\mcI{i}} $ and  $\respI{i}{j} = \resptI{i}{j} \cdot \mcI{i} $, making the content of the response matrix $\resp$ the actual number of expected events.

\subsection{Code and Software}
In HEP the most widespread code is {\scshape root} (and {PyRoot}) with the additional package of {\scshape RooUnfold}.

\section{Understanding unfolding}
\difficulty{2}
The first point of understand how unfolding operate is to deal with the smearing of the distribution.
Suppose to have a fully efficient detector that has just a poor gaussian resolution, as given in the attachments:
\begin{enumerate}
	\item Write the inversion method to unfold the distribution without random fluctuations
	\item Use the inversion method to unfold the distirbution with random fluctuations
	\item discuss the result.
\end{enumerate}
\begin{figure}[Hh!]
	\includegraphics[width=0.49\textwidth]{figs/respt.pdf}
	\includegraphics[width=0.49\textwidth]{figs/reco.pdf}
	\caption{
		\label{fig:exe1}
		Left: smearing matrix ($\respt$) 
		Right: reconstructed distributions with and without random fluctuations.
	}	
\end{figure}
\FloatBarrier


%%%%%%%%%%%%%%%%%%%% BIBLIO %%%%%%%%%%%%%%%%%%%%%%
\FloatBarrier
\nocite{*}
%\bibliographystyle{acm}
\bibliographystyle{acmunsrt}
%\addcontentsline{toc}{section}{\refname}
\bibliography{UnfoldingExe}
%%%%%%%%%%%%%%%%%%
\cleardoublepage

\end{document}
